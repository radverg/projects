\documentclass[11pt,a4paper]{article}
\usepackage[utf8]{inputenc}
\usepackage[czech]{babel}
\usepackage[left=2cm,text={17cm, 24cm},top=3cm]{geometry}
\usepackage{amsfonts}
\usepackage{makeidx}
\usepackage[T1]{fontenc} % fonty
\author{Lukáš Wagner}
\setlength{\parindent}{0pt}

\begin{document}
	\vspace{2cm}
	\begin{center}
			{\Huge IVS - Profiling}\\
		\vspace{0.2cm}
		{\Large
			Testovací tým\\
			\vspace{0.2cm}
			\today
		}
	\end{center}
	\textbf{Výsledek profilovací knihovny:}
	
	\hspace{20pt}15265 function calls in 0.048 seconds
	
	\begin{table}[h]
		\begin{tabular}{|r|r|r|r|r|l|}
			\hline
			\multicolumn{1}{|l|}{\textbf{ncalls}} & \multicolumn{1}{l|}{\textbf{tottime}} & \multicolumn{1}{l|}{\textbf{percall}} & \multicolumn{1}{l|}{\textbf{cumtime}} & \multicolumn{1}{l|}{\textbf{percall}} & \textbf{filename:lineno(function)}    \\ \hline
			1                                     & 0.000                                 & 0.000                                 & 0.048                                 & 0.048                                 & :0(exec)                              \\ 
			6025                                  & 0.009                                 & 0.000                                 & 0.009                                 & 0.000                                 & :0(isinstance)                        \\ 
			1                                     & 0.000                                 & 0.000                                 & 0.000                                 & 0.000                                 & :0(len)                               \\ 
			1                                     & 0.000                                 & 0.000                                 & 0.000                                 & 0.000                                 & :0(print)                             \\ 
			1                                     & 0.000                                 & 0.000                                 & 0.000                                 & 0.000                                 & :0(setprofile)                        \\ 
			200                                   & 0.000                                 & 0.000                                 & 0.000                                 & 0.000                                 & :0(split)                             \\ 
			3                                     & 0.000                                 & 0.000                                 & 0.000                                 & 0.000                                 & :0(utf\_8\_decode)                    \\ 
			1                                     & 0.000                                 & 0.000                                 & 0.048                                 & 0.048                                 & \&lt;string\&gt;:1(\&lt;module\&gt;)  \\ 
			3                                     & 0.000                                 & 0.000                                 & 0.000                                 & 0.000                                 & codecs.py:318(decode)                 \\ 
			1                                     & 0.006                                 & 0.006                                 & 0.048                                 & 0.048                                 & deviation.py:16(run)                  \\ 
			1                                     & 0.000                                 & 0.000                                 & 0.048                                 & 0.048                                 & profile:0(deviation.run())            \\ 
			0                                     & 0.000                                 & 0.000                                 &                                       &                                       & profile:0(profiler)                   \\ 
			2                                     & 0.000                                 & 0.000                                 & 0.000                                 & 0.000                                 & rgg\_mathlib.py:12(sub)               \\ 
			3                                     & 0.000                                 & 0.000                                 & 0.000                                 & 0.000                                 & rgg\_mathlib.py:18(mul)               \\ 
			2                                     & 0.000                                 & 0.000                                 & 0.000                                 & 0.000                                 & rgg\_mathlib.py:24(div)               \\ 
			6018                                  & 0.018                                 & 0.000                                 & 0.026                                 & 0.000                                 & rgg\_mathlib.py:3(\_\_valid\_operand) \\ 
			1001                                  & 0.005                                 & 0.000                                 & 0.014                                 & 0.000                                 & rgg\_mathlib.py:30(pow)               \\ 
			1                                     & 0.000                                 & 0.000                                 & 0.000                                 & 0.000                                 & rgg\_mathlib.py:51(root)              \\ 
			2000                                  & 0.010                                 & 0.000                                 & 0.027                                 & 0.000                                 & rgg\_mathlib.py:6(add)                \\ \hline
		\end{tabular}
	\end{table}
	\vspace{1em}
	
	\textbf{Výsledek odchylky:}
	
	\hspace{20pt} 1.0000000864464458
	\vspace{1.5em}\\
	\textbf{Závěr:}
	
	\setlength{\parindent}{20pt}
	Podle všeho trávíme nejvíce času ověřováním platnosti argumentů (0.018s). To protože ověřování
	platného vstupu je zapotřebí v každém volání knihovní funkce. Kdyby se nám podařilo zefektivnit toto
	ověřování, mělo by to veliký dopad na rychlost naší knihovny.
	
	Zároveň ovšem ověřování platnosti čísel netrvá tak dlouho, jako samy výpočetní funkce. Vidíme, že
	ověření čísla bylo zavoláno 3x častěji než sčítání, ale netrvalo dohromady ani dvakrát tak dlouho. Proto
	by nebylo od věci zaměřit se i na výpočetní funkce, kde už jsme limitováni možnostmi jazyka (Python) a
	jediné řešení by tedy bylo přejít třeba na Cython.
	

\end{document}